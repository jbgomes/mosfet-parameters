%% ou-crc-phd-students-conf.tex
%% V2.0
%% 2015/05/20
%% by Andrea Franceschini and Theo Georgiou
%%
%% BASED ON
%%
%% bare-conf.tex
%% V1.3
%% 2007/01/11
%% by Michael Shell
%% See:
%% http://www.michaelshell.org/
%% for current contact information.
%%
%% This is a skeleton file demonstrating the use of IEEEtran.cls
%% (requires IEEEtran.cls version 1.7 or later) with an IEEE conference paper.
%%
%% Support sites:
%% http://www.michaelshell.org/tex/ieeetran/
%% http://www.ctan.org/tex-archive/macros/latex/contrib/IEEEtran/
%% and
%% http://www.ieee.org/

%%*************************************************************************
%% Legal Notice:
%% This code is offered as-is without any warranty either expressed or
%% implied; without even the implied warranty of MERCHANTABILITY or
%% FITNESS FOR A PARTICULAR PURPOSE! 
%% User assumes all risk.
%% In no event shall IEEE or any contributor to this code be liable for
%% any damages or losses, including, but not limited to, incidental,
%% consequential, or any other damages, resulting from the use or misuse
%% of any information contained here.
%%
%% All comments are the opinions of their respective authors and are not
%% necessarily endorsed by the IEEE.
%%
%% This work is distributed under the LaTeX Project Public License (LPPL)
%% ( http://www.latex-project.org/ ) version 1.3, and may be freely used,
%% distributed and modified. A copy of the LPPL, version 1.3, is included
%% in the base LaTeX documentation of all distributions of LaTeX released
%% 2003/12/01 or later.
%% Retain all contribution notices and credits.
%% ** Modified files should be clearly indicated as such, including  **
%% ** renaming them and changing author support contact information. **
%%
%% File list of work: IEEEtran.cls, IEEEtran_HOWTO.pdf, bare_adv.tex,
%%                    bare_conf.tex, bare_jrnl.tex, bare_jrnl_compsoc.tex
%%*************************************************************************

% *** Authors should verify (and, if needed, correct) their LaTeX system  ***
% *** with the testflow diagnostic prior to trusting their LaTeX platform ***
% *** with production work. IEEE's font choices can trigger bugs that do  ***
% *** not appear when using other class files.                            ***
% The testflow support page is at:
% http://www.michaelshell.org/tex/testflow/



% Note that the a4paper option is mainly intended so that authors in
% countries using A4 can easily print to A4 and see how their papers will
% look in print - the typesetting of the document will not typically be
% affected with changes in paper size (but the bottom and side margins will).
% Use the testflow package mentioned above to verify correct handling of
% both paper sizes by the user's LaTeX system.
%
% Also note that the "draftcls" or "draftclsnofoot", not "draft", option
% should be used if it is desired that the figures are to be displayed in
% draft mode.
%
\documentclass[10pt, conference,a4paper]{IEEEtran}
\usepackage{graphicx}
\usepackage{hyperref}
\usepackage{comment}

%\usepackage[nottoc]{tocbibind}
%\title{Bibliography management: BibTeX}
%\author{Overleaf}

% If IEEEtran.cls has not been installed into the LaTeX system files,
% manually specify the path to it like:
% \documentclass[conference]{../sty/IEEEtran}





% Some very useful LaTeX packages include:
% (uncomment the ones you want to load)


% *** MISC UTILITY PACKAGES ***
%
%\usepackage{ifpdf}
% Heiko Oberdiek's ifpdf.sty is very useful if you need conditional
% compilation based on whether the output is pdf or dvi.
% usage:
% \ifpdf
%   % pdf code
% \else
%   % dvi code
% \fi
% The latest version of ifpdf.sty can be obtained from:
% http://www.ctan.org/tex-archive/macros/latex/contrib/oberdiek/
% Also, note that IEEEtran.cls V1.7 and later provides a builtin
% \ifCLASSINFOpdf conditional that works the same way.
% When switching from latex to pdflatex and vice-versa, the compiler may
% have to be run twice to clear warning/error messages.






% *** CITATION PACKAGES ***
%
%\usepackage{cite}
% cite.sty was written by Donald Arseneau
% V1.6 and later of IEEEtran pre-defines the format of the cite.sty package
% \cite{} output to follow that of IEEE. Loading the cite package will
% result in citation numbers being automatically sorted and properly
% "compressed/ranged". e.g., [1], [9], [2], [7], [5], [6] without using
% cite.sty will become [1], [2], [5]--[7], [9] using cite.sty. cite.sty's
% \cite will automatically add leading space, if needed. Use cite.sty's
% noadjust option (cite.sty V3.8 and later) if you want to turn this off.
% cite.sty is already installed on most LaTeX systems. Be sure and use
% version 4.0 (2003-05-27) and later if using hyperref.sty. cite.sty does
% not currently provide for hyperlinked citations.
% The latest version can be obtained at:
% http://www.ctan.org/tex-archive/macros/latex/contrib/cite/
% The documentation is contained in the cite.sty file itself.






% *** GRAPHICS RELATED PACKAGES ***
%
\ifCLASSINFOpdf
  % \usepackage[pdftex]{graphicx}
  % declare the path(s) where your graphic files are
  % \graphicspath{{../pdf/}{../jpeg/}}
  % and their extensions so you won't have to specify these with
  % every instance of \includegraphics
  % \DeclareGraphicsExtensions{.pdf,.jpeg,.png}
\else
  % or other class option (dvipsone, dvipdf, if not using dvips). graphicx
  % will default to the driver specified in the system graphics.cfg if no
  % driver is specified.
  % \usepackage[dvips]{graphicx}
  % declare the path(s) where your graphic files are
  % \graphicspath{{../eps/}}
  % and their extensions so you won't have to specify these with
  % every instance of \includegraphics
  % \DeclareGraphicsExtensions{.eps}
\fi
% graphicx was written by David Carlisle and Sebastian Rahtz. It is
% required if you want graphics, photos, etc. graphicx.sty is already
% installed on most LaTeX systems. The latest version and documentation can
% be obtained at: 
% http://www.ctan.org/tex-archive/macros/latex/required/graphics/
% Another good source of documentation is "Using Imported Graphics in
% LaTeX2e" by Keith Reckdahl which can be found as epslatex.ps or
% epslatex.pdf at: http://www.ctan.org/tex-archive/info/
%
% latex, and pdflatex in dvi mode, support graphics in encapsulated
% postscript (.eps) format. pdflatex in pdf mode supports graphics
% in .pdf, .jpeg, .png and .mps (metapost) formats. Users should ensure
% that all non-photo figures use a vector format (.eps, .pdf, .mps) and
% not a bitmapped formats (.jpeg, .png). IEEE frowns on bitmapped formats
% which can result in "jaggedy"/blurry rendering of lines and letters as
% well as large increases in file sizes.
%
% You can find documentation about the pdfTeX application at:
% http://www.tug.org/applications/pdftex





% *** MATH PACKAGES ***
%
%\usepackage[cmex10]{amsmath}
% A popular package from the American Mathematical Society that provides
% many useful and powerful commands for dealing with mathematics. If using
% it, be sure to load this package with the cmex10 option to ensure that
% only type 1 fonts will utilized at all point sizes. Without this option,
% it is possible that some math symbols, particularly those within
% footnotes, will be rendered in bitmap form which will result in a
% document that can not be IEEE Xplore compliant!
%
% Also, note that the amsmath package sets \interdisplaylinepenalty to 10000
% thus preventing page breaks from occurring within multiline equations. Use:
%\interdisplaylinepenalty=2500
% after loading amsmath to restore such page breaks as IEEEtran.cls normally
% does. amsmath.sty is already installed on most LaTeX systems. The latest
% version and documentation can be obtained at:
% http://www.ctan.org/tex-archive/macros/latex/required/amslatex/math/





% *** SPECIALIZED LIST PACKAGES ***
%
%\usepackage{algorithmic}
% algorithmic.sty was written by Peter Williams and Rogerio Brito.
% This package provides an algorithmic environment fo describing algorithms.
% You can use the algorithmic environment in-text or within a figure
% environment to provide for a floating algorithm. Do NOT use the algorithm
% floating environment provided by algorithm.sty (by the same authors) or
% algorithm2e.sty (by Christophe Fiorio) as IEEE does not use dedicated
% algorithm float types and packages that provide these will not provide
% correct IEEE style captions. The latest version and documentation of
% algorithmic.sty can be obtained at:
% http://www.ctan.org/tex-archive/macros/latex/contrib/algorithms/
% There is also a support site at:
% http://algorithms.berlios.de/index.html
% Also of interest may be the (relatively newer and more customizable)
% algorithmicx.sty package by Szasz Janos:
% http://www.ctan.org/tex-archive/macros/latex/contrib/algorithmicx/




% *** ALIGNMENT PACKAGES ***
%
%\usepackage{array}
% Frank Mittelbach's and David Carlisle's array.sty patches and improves
% the standard LaTeX2e array and tabular environments to provide better
% appearance and additional user controls. As the default LaTeX2e table
% generation code is lacking to the point of almost being broken with
% respect to the quality of the end results, all users are strongly
% advised to use an enhanced (at the very least that provided by array.sty)
% set of table tools. array.sty is already installed on most systems. The
% latest version and documentation can be obtained at:
% http://www.ctan.org/tex-archive/macros/latex/required/tools/


%\usepackage{mdwmath}
%\usepackage{mdwtab}
% Also highly recommended is Mark Wooding's extremely powerful MDW tools,
% especially mdwmath.sty and mdwtab.sty which are used to format equations
% and tables, respectively. The MDWtools set is already installed on most
% LaTeX systems. The lastest version and documentation is available at:
% http://www.ctan.org/tex-archive/macros/latex/contrib/mdwtools/


% IEEEtran contains the IEEEeqnarray family of commands that can be used to
% generate multiline equations as well as matrices, tables, etc., of high
% quality.


%\usepackage{eqparbox}
% Also of notable interest is Scott Pakin's eqparbox package for creating
% (automatically sized) equal width boxes - aka "natural width parboxes".
% Available at:
% http://www.ctan.org/tex-archive/macros/latex/contrib/eqparbox/





% *** SUBFIGURE PACKAGES ***
%\usepackage[tight,footnotesize]{subfigure}
% subfigure.sty was written by Steven Douglas Cochran. This package makes it
% easy to put subfigures in your figures. e.g., "Figure 1a and 1b". For IEEE
% work, it is a good idea to load it with the tight package option to reduce
% the amount of white space around the subfigures. subfigure.sty is already
% installed on most LaTeX systems. The latest version and documentation can
% be obtained at:
% http://www.ctan.org/tex-archive/obsolete/macros/latex/contrib/subfigure/
% subfigure.sty has been superceeded by subfig.sty.



%\usepackage[caption=false]{caption}
%\usepackage[font=footnotesize]{subfig}
% subfig.sty, also written by Steven Douglas Cochran, is the modern
% replacement for subfigure.sty. However, subfig.sty requires and
% automatically loads Axel Sommerfeldt's caption.sty which will override
% IEEEtran.cls handling of captions and this will result in nonIEEE style
% figure/table captions. To prevent this problem, be sure and preload
% caption.sty with its "caption=false" package option. This is will preserve
% IEEEtran.cls handing of captions. Version 1.3 (2005/06/28) and later 
% (recommended due to many improvements over 1.2) of subfig.sty supports
% the caption=false option directly:
%\usepackage[caption=false,font=footnotesize]{subfig}
%
% The latest version and documentation can be obtained at:
% http://www.ctan.org/tex-archive/macros/latex/contrib/subfig/
% The latest version and documentation of caption.sty can be obtained at:
% http://www.ctan.org/tex-archive/macros/latex/contrib/caption/




% *** FLOAT PACKAGES ***
%
%\usepackage{fixltx2e}
% fixltx2e, the successor to the earlier fix2col.sty, was written by
% Frank Mittelbach and David Carlisle. This package corrects a few problems
% in the LaTeX2e kernel, the most notable of which is that in current
% LaTeX2e releases, the ordering of single and double column floats is not
% guaranteed to be preserved. Thus, an unpatched LaTeX2e can allow a
% single column figure to be placed prior to an earlier double column
% figure. The latest version and documentation can be found at:
% http://www.ctan.org/tex-archive/macros/latex/base/



%\usepackage{stfloats}
% stfloats.sty was written by Sigitas Tolusis. This package gives LaTeX2e
% the ability to do double column floats at the bottom of the page as well
% as the top. (e.g., "\begin{figure*}[!b]" is not normally possible in
% LaTeX2e). It also provides a command:
%\fnbelowfloat
% to enable the placement of footnotes below bottom floats (the standard
% LaTeX2e kernel puts them above bottom floats). This is an invasive package
% which rewrites many portions of the LaTeX2e float routines. It may not work
% with other packages that modify the LaTeX2e float routines. The latest
% version and documentation can be obtained at:
% http://www.ctan.org/tex-archive/macros/latex/contrib/sttools/
% Documentation is contained in the stfloats.sty comments as well as in the
% presfull.pdf file. Do not use the stfloats baselinefloat ability as IEEE
% does not allow \baselineskip to stretch. Authors submitting work to the
% IEEE should note that IEEE rarely uses double column equations and
% that authors should try to avoid such use. Do not be tempted to use the
% cuted.sty or midfloat.sty packages (also by Sigitas Tolusis) as IEEE does
% not format its papers in such ways.





% *** PDF, URL AND HYPERLINK PACKAGES ***
%
%\usepackage{url}
% url.sty was written by Donald Arseneau. It provides better support for
% handling and breaking URLs. url.sty is already installed on most LaTeX
% systems. The latest version can be obtained at:
% http://www.ctan.org/tex-archive/macros/latex/contrib/misc/
% Read the url.sty source comments for usage information. Basically,
% \url{my_url_here}.





% *** Do not adjust lengths that control margins, column widths, etc. ***
% *** Do not use packages that alter fonts (such as pslatex).         ***
% There should be no need to do such things with IEEEtran.cls V1.6 and later.
% (Unless specifically asked to do so by the journal or conference you plan
% to submit to, of course. )


% correct bad hyphenation here
\hyphenation{op-tical net-works semi-conduc-tor}


\begin{document}
%
% paper title
%\title{Paper Title} 
\title{	MOS Parameters Extraction \\  

\vspace{0.3\baselineskip} \Large{}} % Comment this and use the one above if you don't need a subtitle

% author names and affiliations
\author{ 
\IEEEauthorblockN{Julia Gomes}
\IEEEauthorblockA{juliatb.gomes@gmail.com}
}

% make the title area
\maketitle

%\noindent
%\textbf{Supervisors name/s:}\\
%\textbf{Starting date:} dd/mm/yyyy (Full-Time / Part-Time)












%%%%%%%%%%%%%%%%% INTRODUÇÃO
\section{Introduction}
O Mosfet é um dispositivo eletrônico que é utilizado em circuitos integrados devido suas características físicas e elétricas. Neste relatório será analisado o comportamento do MOS a partir de fundamentação e simulações.




%%%%%%%%%%%%%%%%% TEORIA
\section{Basic MOS Device Physics}

\subsection{Threshold Voltage}
Quando uma tensão é inserida no terminal do gate em um transistor tipo N [Fig. \ref{fig:Threshold}(a)], buracos em p-substrate são repelidos da área do gate e elétrons são atraídos, formando uma região de depleção [Fig. \ref{fig:Threshold}(b)]. Nessa condição ainda não há disponibilidade de portadores de carga \cite{razavi2005design}. 

\begin{figure}[h]
    \centering
    \includegraphics[width=1\linewidth]{Figures/threshold_voltage.png}
    \caption{(a) A MOSFET driven by a gate voltage; (b) formation of depletion region; (c) onset of inversion; (d) formation of inversion layer}
    \label{fig:Threshold}
\end{figure}

Com o aumento de $V_G$, a largura da região de depleção aumenta assim como o potencial na interface óxido-silício. A estrutura é similar a um divisor de tensão, consistindo de dois capacitores em série: gate-oxide capacitor and the depletion-region capacitor [Fig. \ref{fig:Threshold}(c)]. Quando a interface assume um valor suficientemente positivo, ocorre o fluxo de corrente \cite{razavi2005design}.

Então, um canal de portadores de carga é criado sob oxido do gate entre S e D, e o transistor é tornado on. A interface é considerada invertida e por isso o canal também é chamado de camada de inversão. O valor de $V_G$ para que isso ocorra é chamado de tensão de threshold, $V_{TH}$. Na realidade, o fenômeno turn-on é uma função gradual da tensão de gate \cite{razavi2005design}. 

Se $V_G$ continua a aumentar, a carga na região de depleção permanece relativamente constante enquanto a densidade de carga no canal continua a aumentar, provendo uma grande corrente de S para D \cite{razavi2005design}.

Similarmente ocorre para dispositivos PMOS, mas com as polaridades invertidas \ref{fig:vth_pmos}. A tensão gate-source precisa estar suficientemente negativa para a ocorrência de uma camada de inversão consistindo de buracos na interface oxide-silicon, provendo um caminho de condução entre S e D \cite{razavi2005design}. 

\begin{figure}[h]
    \centering
    \includegraphics[width=1\linewidth]{Figures/vth_pmos.png}
    \caption{Formation of inversion layer in a PFET}
    \label{fig:vth_pmos}
\end{figure}

\subsection{Derivation of I/V Characteristics}
\begin{comment}
O transistor MOS é um dispositivo de quatro terminais: gate, drain, source and body. As Figuras \ref{fig:sch} e \ref{fig:lay} mostram a representação de schematic e layout do Mosfet, respectivamente.

\begin{figure}[h]
    \centering
    \includegraphics[width=1\linewidth]{Figures/mos_symbol.png}
    \caption{MOS Symbol}
    \label{fig:sch}
\end{figure}

\begin{figure}[h]
    \centering
    \includegraphics[width=1\linewidth]{Figures/mos_lay.png}
    \caption{Structure of a MOS device}
    \label{fig:lay}
\end{figure}
\end{comment}

O MOSFET é uma função de 5 variáveis: três tensões ($V_{GS}$, $V_{DS}$ e $V_{SB}$) e dois parâmetros de tamanho ($L$ e $W$). Esse dispositivo pode operar em três regiões de operação. As equações que serão listadas nas próximas seções são oriundas do transistivo NMOS. 

As equações para o dispositivo PMOS são escritas de forma similar, mas com um sinal negativo na frente porque é considerando que os buracos fluem do source para o drain. Além disso, $V_{GS}$, $V_{DS}$, $V_{TH}$ e $V_{GS}-V_{TH}$ são negativos para um transistor PMOS que está ligado. Como a mobilidade dos buracos é cerca de metade da mobilidade dos elétrons, os dispositivos PMOS sofrem de menor capacidade de current drive \cite{razavi2005design}.

%$V_{GS}$ é uma variável primária e $L$ é secundária. Geralmente é configurado um valor de referência para $W$ \cite{youtube_gm/Id}. 

\subsubsection{Cut off Region}
  
$V_{GS} < V_{TH}$: $I_D = 0$.

O transistor está turn-off.
    
\subsubsection{Triodo or Linear Region}
    
    $V_{DS} \leq V_{GS} - V_{TH}$:
    \begin{equation}        
    I_D = \mu_n C_{ox} \frac{W}{L} \left [(V_{GS} - V_{TH})V_{DS} - \frac{1}{2}V_{DS}^2 \right]
\label{eq:triode}
    \end{equation}
    
    A capaciadade de corrente do dispositivo aumenta com $V_{GS}$. O pico da parábola ($\frac{\Delta I_D}{\Delta V_{DS}}$) da Figura \ref{fig:triode} ocorre na tensão de overdrive ($V_{ov}$), $V_{DS} = V_{GS} - V_{TH}$.
    
    \begin{figure}[h]
    \centering
    \includegraphics[width=1\linewidth]{Figures/triode.png}
    \caption{$I_D$ vs $V_{GS}$ in the triode region}
    \label{fig:triode}
    \end{figure}

    Quando $V_{DS} \ll 2(V_{GS} - V_{TH})$:
    \begin{equation}
    I_D \approx \mu_n C_{ox} \frac{W}{L} (V_{GS} - V_{TH})V_{DS}
    \label{eq:deep_triode}
  \end{equation}  
  
    Quando $V_{DS}$ é muito pequeno, o caminho entre source e drain pode ser representado por um resistor linear igual a
    \begin{equation}
        R_{on} = \frac{1}{\mu_n C_{ox} \frac{W}{L} (V_{GS} - V_{TH})}
    \end{equation}
    
    
\subsubsection{Saturation Region}
$V_{DS} > V_{ov}$:
    \begin{equation}
    I_D = \frac{1}{2} \mu_n C_{ox} \frac{W}{L'} (V_{GS} - V_{TH})^2
    \label{eq:saturação}
\end{equation}
em que $L'$ é o ponto em que $Q_d$ cai para zero (pinch-off).

Quando a tensão de drain é maior que a tensão de overdrive, a corrente tende a se manter constante [Fig. \ref{fig:saturation}]. Se $V_{DS}$ é significativamente maior que $V_{GS} - V_{TH}$, então a camada de inversão para em $x \leq L$ [Fig. \ref{fig:pinch-off}], e o canal é considerado pinched off \cite{razavi2005design}.

\begin{figure}[h]
\centering
\includegraphics[width=1\linewidth]{Figures/saturation.png}
    \caption{$I_D$ vs $V_{GS}$ in the saturation region}
    \label{fig:saturation}
\end{figure}

Com $V_{DS}$ crescendo ainda mais, o ponto em que $Q_d$ é igual a zero move gradualmente para o source. Então, em algum ponto ao longo do canal, a diferença de potencial entre gate e a interface oxide-silicon não é suficiente para suportar a camada de inversão \cite{razavi2005design}.

\begin{figure}[h]
\centering
\includegraphics[width=1\linewidth]{Figures/pinch-off.png}
    \caption{Pinch-off behavior}
    \label{fig:pinch-off}
\end{figure}

Mesmo com pinch-off, o dispositivo continua conduzindo corrente. Quando os elétrons se aproximam do ponto pinch-off (onde $Q_d \rightarrow 0 $), sua velocidade aumenta ($v = I/Q_d$), e passando por esse ponto eles disparam através da região de depleção perto da junção de dreno e chegam ao terminal de dreno \cite{razavi2005design}.

\subsection{MOS Transconductance}
A transcondutânica do MOSFET é a capacidade do dispositivo de converter tensão de overdrive de gate-source em corrente de drain. $g_m$ é representado em siemens (S) e na região de saturação pode ser dado pelo inverso de $R_{on}$ na região de triodo profundo \cite{razavi2005design}. Essa quantidade, para $V_{DS}$ constante, é expressa como

\begin{equation}
    g_m = \frac{\partial I_{D}}{\partial V_{GS}}  \label{eq:transconductance}
\end{equation}

\begin{equation}
     = \mu_n C_{ox} \frac{W}{L} (V_{GS} - V_{TH}) \label{eq:transconductance1}
\end{equation}

\begin{equation}
     = \sqrt{2 \mu_n C_{ox} \frac{W}{L} I_D} \label{eq:transconductance2}
\end{equation}

\begin{equation}
     = \frac{2 I_D}{V_{GS} - V_{TH}} \label{eq:transconductance2}
\end{equation}

A relação $g_m/I_D = 2/(V_{GS} - V_{TH})$ é uma métrica de eficiência que mostra quanta transcondutância é obtida por unidade de corrente, importante para o design de circuitos analógicos eficientes. Essa eficiência diminui com o aumento de $V_{GS}$, refletindo a natureza não linear e os limites físicos do MOSFET \cite{youtube_gm/Id}.

\begin{comment}
\begin{equation}
    r_o = \frac{1}{g_m}
    \label{eq:res_out}
\end{equation}

Tensão de drain
\begin{equation}
    V_D = -g_m V_{GS} r_o
\end{equation}

Ganho intrínseco do transistor
\begin{equation}
    V_D = \left | A_v \right | = \frac{V_D}{V_{GS}} = g_m r_o
\end{equation}
\end{comment}

\subsection{Second-order effects}

\subsubsection{Body Effect} À medida que $V_B$ se torna mais negativo, mais buracos são atraídos para a conexão do substrato, deixando uma carga negativa maior para trás; ou seja, a região de depleção se torna
mais ampla. A tensão de limiar é uma função da carga total na região de depleção
porque a carga do gate deve espelhar Qd antes que uma camada de inversão seja formada. Assim, conforme $V_B$ cai e $Q_d$ aumenta, $V_{TH}$ também aumenta. Esse fenômeno é chamado de body effect e $\gamma$ é o body effect coefficient.

\subsubsection{Channel-Length Modulation} Na análise de pinch-off do canal, notamos que o comprimento atual do canal gradualmente diminui com a queda da diferença de potencial entre gate e drain. $L'$ em \ref{eq:saturação} é uma função de $V_{DS}$. Esse efeito é chamado de channel-length modulation, onde $\lambda$ é o channel-length modulation coefficient, que apresenta valores pequenos para longos canais \cite{razavi2005design}.

\subsection{Subthreshold Conduction}
Quando $V_{GS} < V_{TH}$, o transistor não desliga abruptamente, há uma fraca inversão de camada e a corrente de drain apresenta uma dependência exponencial a $V_{GS}$ [Eq. \ref{eq.subthreshold_Id}]. 

\begin{equation}
    I_D = I_0 exp \frac{V_{GS}}{\xi V_T}
    \label{eq.subthreshold_Id}
\end{equation}

 onde $I_0$ é proporcional a $\frac{W}{L}$, $\xi > 1$ é um fator de não linearidade, e $V_T = \frac{kT}{q}$.

\subsection{MOS Small-Signal Model}
Se a perturbação nas condições de polarização é pequena, um small-Signal model [Fig. \ref{fig:small-model}(a)] aproximando o modelo de large-signal ao redor do ponto de operação pode ajudar na simplificação dos cálculos.

\begin{figure}[h]
\centering
\includegraphics[width=1\linewidth]{Figures/small-model.png}
    \caption{Pinch-off behavior}
    \label{fig:small-model}
\end{figure}

Considerando a channel-length modulation, a corrente de drain também varia com tensão drain-source. Esse efeito pode ser modelado por um fonte de corrente de voltage-dependent [Fig. \ref{fig:small-model}(b)], mas uma fonte de corrente cujo valor depende linearmente da voltagem através dela é equivalente a um resistor linear [Fig. \ref{fig:small-model}(c)]. Esse resistor de saída afeta o desempenho de muitos circuitos analógicos e pode ser dado por

\begin{equation}
    r_o = \frac{\partial V_{DS}}{\partial I_D}
\end{equation}

\begin{equation}
    = \frac{1}{\partial I_D / \partial V_{DS}}
\end{equation}

\begin{equation}
    = \frac{1}{\frac{1}{2} \mu_n C_{ox} \frac{W}{L} (V_{GS} - V_{TH})^2 \lambda}
\end{equation}

O potencial em bulk influencia a threshold voltage e, portanto, o overdrive gate-source. Com todos os outros terminais mantidos em uma tensão constante, a corrente de dreno é uma função da tensão em bulk. Ou seja, o bulk se comporta como um segundo gate. Modelando essa dependência
por uma fonte de corrente conectada entre D e S [Fig. \ref{fig:small-model}(d)], escrevemos o valor como $g_{mb}V_{bs}$, onde
$g_{mb} = \partial I_D / \partial V_{BS}$. Na região de saturação, $g_{mb}$ pode ser expressa como

\begin{equation}
    g_m = \frac{\partial I_D}{\partial V_{BS}}
\end{equation}

\begin{equation}
    = \mu_n C_{ox} \frac{W}{L} (V_{GS} - V_{TH}) \left (- \frac{\partial V_{TH}}{\partial V_{BS}} \right)
\end{equation}

\begin{equation}
    = g_m \left (\frac{\partial V_{TH}}{\partial V_{SB}} \right)
\end{equation}

\begin{equation}
    = g_m \frac{\gamma}{2 \sqrt{2 \Phi_F + V_{SB}}} 
\end{equation}

\begin{equation}
    = \eta g_m 
\end{equation}

onde $\eta = g_{bm}/g_m$ e é tipicamente em torno de $0.25$.

\begin{comment}
\begin{figure}[h]
    \centering
    \includegraphics[width=1\linewidth]{Figures/small_signal.png}
    \caption{MOS Small-Signal Model}
    \label{fig:small-signal}
\end{figure}

No slide fala que a lei quadrática falha pra descrever inversão fraca. Acho que nesse slide abaixo tá dizendo que a lei quadrática pode ser útil em transistores de canal longo porque os efeitos de canal curto são mais proeminentes.

Ler sobre efeito de canal curto. Porque o gráfico mostra que L tá influenciando em ID, mas no slide tá dizendo que L não influencia fracamente em ID. Ver se o gráfico que plotei também apresenta isso.
\end{comment}

\begin{comment}
\subsection{Long-Channel Versus Short-Channel Devices}

Efeitos de canal curto: saturação da velocidade que pode ser gerada porque a corrente de drain tem uma relação linear com vgs, e degradação na mobilidade. 

Outras coisas também que não entendi muito bem em relação a gm saturar com vgs-vth e lei quadrática falha pra descrever inversão fraca. 
Mosfets de canal curto são geralmente muito rápidos, mas não existe modelo simples.
\end{comment}








%%%%%%%%%%%%%%%%% SIMULATION
\section{Simulation Results}
A extração dos parâmetros do Mosfet foi realizada a partir de simulações no software Cadence Virtuoso. A tecnologia utilizada foi o gpdk de $45$ nm disponibilizada pela Cadence. 

Tanto o transistor PMOS quanto o NMOS foram simulados [Fig. \ref{fig:tb}]. Fontes dc foram inseridas nos terminais de drain e gate para os dois transsitores. No dispositivo PMOS, os terminais source e body foram conectados ao vdd e no transistor NMOS, esses terminais foram ligados ao vss.

\begin{figure}[h]
    \centering
    \includegraphics[width=1\linewidth]{Figures/tb.png}
    \caption{Testbench}
    \label{fig:tb}
\end{figure}

\subsection{Drain current}
A tensão gate-source controla a corrente que flui no drain. A Figura \ref{fig:IdxVgs} mostra esse comportamento para os transistores PMOS e NMOS obtidos a partir de uma simulação dc. Foram considerados dois tamanhos para o transistor: $1$ um representando canal longo e $45$ nm representando canal curto. O valor de $W$ foi determinado como $10$ vezes o valor de $L$, $V_{DS} = 1.8$ V, $vdd = 0$ V e $V_{GS}$ variando de $0$ a $2.5$ V. A simulação mostra que com o canal mais curto o efeito de $V_{GS}$ é menor no aumento de $I_D$. 

\begin{figure}[h]
    \centering
    \includegraphics[width=1\linewidth]{Figures/IdxVgs.png}
    \caption{$I_D$ vs $V_{GS}$ to $L =$ [$45$ nm $1$ um] to PMOS and NMOS}
    \label{fig:IdxVgs}
\end{figure}

Quando $V_{GS} < V_{TH}$, o transistor está desligado e a corrente é quase nula. Quando $V_{GS} > V_{TH}$ e $V_{DS}$ é pequeno, o transistor se comporta como um resistor controlado por $V_{GS}$. Quando $V_{GS} > V_{TH}$ e $V_{DS}$ é grande o suficiente para que o transistor esteja na saturação, $I_D$ é praticamente constante em relação a $V_{DS}$ e varia com relação a $V_{GS}$.

A Figura \ref{fig:IdxVds} mostra a relação entre corrente de dreno e tensão de drain-source para os transistores PMOS e NMOS obtidas a partir de uma simulação dc. O tamanho do transistor foi considerado $1$ um, $V_{GS}$ assumiu o valor $1$ V e $1.8$ V para análise do seu efeito na curva, $vdd = 0$ e $V_{DS}$ variando de $0$ a $2.5$ V. 

\begin{figure}[h]
    \centering
    \includegraphics[width=1\linewidth]{Figures/IdxVds.png}
    \caption{$I_D$ vs $V_{DS}$ to $V_{GS} =$ [$1$ V $1.8$ V] to PMOS and NMOS}
    \label{fig:IdxVds}
\end{figure}

Também foi analisado o comportamento da corrente de drain em relação a tensão drain-source para canal curto e canal longo. A Figura \ref{fig:IdxVdsshortL} mostra o resultado dessa simulação. Os mesmos parâmetros foram utilizados, mas deixando $V_{GS}$ apenas assumindo o valor de $1$ V.

\begin{figure}[h]
    \centering
    \includegraphics[width=1\linewidth]{Figures/IdxVds_shortL.png}
    \caption{$I_D$ vs $V_{DS}$ to $L =$ [$45$ nm $1$ um] to PMOS and NMOS}
    \label{fig:IdxVdsshortL}
\end{figure}

%Basicamente, o transistor funciona como uma fonte de corrente controlada por tensão. Essa função habilita desde aplicações de amplificação e processamento de sinal até lógica digital e circuitos de memória.

Idealmente, a tensão $V_{DS}$ não afeta $I_D$ porque o Mosfet é uma fonte de corrente controlada por tensão para $V_{DS} > V_{Dsat}$. Na prática, aumentando $V_{DS}$, $I_D$ também aumenta. 

%Efeito $V_{DS}$ modelado por $r_o = \frac{V_A}{I_D} = \frac{1}{\lambda I_D}$. 

% $V_{GS}$ é a primeira tensão controlando o comportamento do dispositivo. Em analog IC, nós geralmente configuramos a corrente de polarização (ID) com relação a polaridade da tensão ($V_GS$).

%Efeitos de Canal Curto: Subthreshold Swing: Mudança rápida na corrente de drain com pequenas variações em VGS. Essa não entendi porque pelo que vi no gráfico a corrente de drain responde mais a Vgs quando o canal é maior.

%Capacidades Parasitárias: Aumento das capacidades parasitárias e efeitos de acoplamento entre gate e canal.

%Efeitos de Canal Longo: Mobilidade de Portadores: A mobilidade de portadores tende a ser mais estável e menos afetada por efeitos de campo elétrico. Resistência do Canal: A resistência do canal é predominantemente linear em função de VDS.

%Id é sempre proporcional w, com aumento de w maior espaço para a passagem de elétrons.

%Efeito de canal curto: Drenagem Induzida por Difusão de Substrato (DIBL): A diminuição da tensão de limiar Vth com o aumento da tensão de drain (VDS). Isso acontece tanto no canal curto quanto longo, mas no canal curto é mais considerável.

Figure \ref{fig:vth} mostra que $V_{TH}$ apresenta um menor valor com canal longo. Além disso, tende a diminuir com o aumento de $V_{DS}$.

\begin{figure}[h]
    \centering
    \includegraphics[width=1\linewidth]{Figures/vth.png}
    \caption{$V_{TH}$ vs $V_{DS}$ to $L =$ [$45$ nm $1$ um] to PMOS and NMOS}
    \label{fig:vth}
\end{figure}

%Para casos de canl curto é necessário um $V_{GS}$ maior para que o transistor ligue.

%A transcondutância é a capacidade do dispositivo de converter tensão em corrente. A Equação \ref{eq:transconductance} é a fórmula da transcondutância do transistor MOS. Se o objetivo é obter transcondutância, deve ser entregue corrente para o dispositivo.

A Figura \ref{fig:gm} mostra a curva da transcondutânica com relação a tensão gate-source. Se o objetivo é obter transcondutância, deve ser entregue corrente para o dispositivo \cite{youtube_gm/Id}. 

%A transcondutância controla velocidade e ruído. O ruído térmico tanto do mosfet quanto do tbj são inversamente proporcionais a $g_m$.

\begin{figure}[h]
    \centering
    \includegraphics[width=1\linewidth]{Figures/gm.png}
    \caption{$g_m/I_D$ vs $V_{GS}$ to $L =$ [$45$ nm $1$ um] to PMOS and NMOS}
    \label{fig:gm}
\end{figure}

%Decrescimento de gm com VGS: 

Para valores baixos de $V_{GS}$, o transistor está na região de subthreshold, onde a corrente de drain cresce exponencialmente com $V_{GS}$. Neste regime, $g_m$ é relativamente alto. À medida que $V_{GS}$ aumenta além do limiar (Vth), o transistor entra na região de operação linear e depois na região de saturação. Em níveis muito altos de VGS, a mobilidade dos portadores pode diminuir devido ao efeito de velocidade de saturação, resultando em uma diminuição de $g_m$.

%Gráfico da transcondutância com relação a corrente de drain também é interessante ser inserido. 

Para canal curto, $V_{DS}$ apresenta uma forte influência em $g_m$, até atingir um ponto de saturação em que canal curto e canal longo apresentam praticamente o mesmo comportamento [Fig. \ref{fig:gm_vds}].

\begin{figure}[h]
    \centering
    \includegraphics[width=1\linewidth]{Figures/gm_vds.png}
    \caption{$g_m/I_D$ vs $V_{DS}$ to $L =$ [$45$ nm $1$ um] to PMOS and NMOS}
    \label{fig:gm_vds}
\end{figure}

A condutância de saída ($g_{ds}$) e é a derivada da corrente de drain em relação a tensão de drain, $g_{ds} = \frac{\partial I_D}{\partial V_{DS}}$, mantendo $V_{GS}$ constante, e está relacionada a resistência de saída do transistor [Fig. \ref{fig:gdsxvds}]. A razão $\frac{g_m}{g_{ds}}$ é conhecida como ganho intrínseco do transistor e é um indicador importante do desempenho do dispositivo, principamente amplificadores.

\begin{figure}[h]
    \centering
    \includegraphics[width=1\linewidth]{Figures/gdsxvds.png}
    \caption{$g_{ds}$ vs $V_{DS}$ to $L =$ [$45$ nm $1$ um] to PMOS and NMOS}
    \label{fig:gdsxvds}
\end{figure}

\begin{comment}
Quando um MOSFET está na região de saturação, a corrente de drain $I_D$ não é totalmente independente de $V_{DS}$. Em vez disso, há uma leve dependência devido ao efeito de modulação da largura do canal, que é capturado pela tensão de Early. O efeito de Early é causado pela variação da largura do canal efetivo conforme $V_{DS}$ aumenta. Isso altera a resistência de saída do transistor e, portanto, afeta a corrente de drain \cite{youtube_gm/Id}.

A resistência de saída ($r_o$) de um MOSFET na região de saturação é inversamente proporcional a condutância ($g_{ds}$) e pode ser relacionada à tensão de Early:

\begin{equation}
    r_o = \frac{1}{g_{ds}} = \frac{V_A}{I_D}
\end{equation}

Corrente de drain na região de saturação pode ser aproximada por:

 \begin{equation}
    I_D = I_{D0} \left (1+\frac{V_{DS}}{V_{A}} \right)
\end{equation}
 
Tentar plotar a resistência passando uma reta no gráfico da corrente em relação a $V_{DS}$

$V_A$ aumenta quando aumentamos $V_{DS}$ e também quando L aumenta. $r_o$ aumenta quando aprofundamos na saturação.

\begin{figure}[h]
    \centering
    \includegraphics[width=1\linewidth]{Figures/roxvds.png}
    \caption{$r_o$ vs $V_{DS}$}
    \label{fig:roxvds}
\end{figure}

\subsection{Ganho}
\begin{equation}
    v_{out} = -g_m V_{GS} r_o 
\end{equation}

Ganho intrínseco do transistor
\begin{equation}
    \left | A_v \right | = \frac{V_{DS}}{V_{GS}} = g_m r_o
\end{equation}

Ganho é a relação da tensão de saída com a tensão de entrada. eu inseri logo o que é saída e entrada do mos, espero que faça sentido.

Na perspectiva de ganho e ruído, um largo $g_m/I_D$ deve ser bom para o NMOS de um inversor porque contribui para o seu ganho. Mas um pequeno $g_m/I_D$ é melhor para o PMOS porque $g_m$ não contribui para o ganho e há uma alta tensão de Early.
\end{comment}


\begin{comment}  

Inserir quando tiver desenvolvendo amplificadores de 2 estágios
Ruído Térmico
M1 é o NMOS e M2 é o PMOS
\begin{equation}
    v_{n,in}^2(f) \approx \frac{4kT\lambda}{g_{m1}} \left (1 + \frac{g_{m2}}{g_{m1}} \right)
\end{equation}
\end{comment}




% essa largura de banda será utilizado mais em casos de amplificador. então passará por momento
\begin{comment}
\subsection{Largura de banda}
Acho que a equação \ref{eq:bw} é da largura de banda.

\begin{equation}
    BW = \frac{\omega_p}{2\pi} = \frac{\omega_p}{2\pi r_o C_L}
    \label{eq:bw}
\end{equation}

Já a equação \ref{eq:gbw}

\begin{equation}
    GBW = \left | A_v \right | B_w = f_u = \frac{g_m}{2 \pi C_L}
    \label{eq:gbw}
\end{equation}
\end{comment}

%\subsection{Efeito de Body}
%A tensão $V_{SB}$ causa efeito de body, aumentar vsb aumneta vth. colocar gráfico de vth em função de vsb pra provar isso. Geralmente, VSB não é um grau-de-liberdade do designer, mas é imposto pela topologia

%Para dispositivos em um poço well, o body pode ser vinculado à fonte. Mas área extra, capacitância extra, acoplamento entre S e D, e talvez ruído extra.

%vou dar um tempo nesse porque até então não sei como seria esse plot

\section{Future Work}
Design and simulations of one-stage amplifiers. 

%%%%%%%%%%%%%%%%% CONCLUSION
\section{Conclusion}
Menor L permite menor área, menor capacitância e maior velocidade ($f_T = \frac{g_m}{2 \pi C_{gg}}$). Entretanto, com L maiores é possível obter maior ganho, menor variação de casamento e baixo ruído de flicker.

%Resumo dos Resultados: Recapitulação dos principais achados da simulação.

%Contribuições e Aprendizados: Discussão sobre o que foi aprendido com a simulação e sua contribuição para o entendimento do circuito ou fenômeno.

%\section{Acknowledgment}

%\include{Reference.bib}
\bibliographystyle{acm}
\bibliography{reference.bib}  

% An example of a floating figure using the graphicx package.
% Note that \label must occur AFTER (or within) \caption.
% For figures, \caption should occur after the \includegraphics.
% Note that IEEEtran v1.7 and later has special internal code that
% is designed to preserve the operation of \label within \caption
% even when the captionsoff option is in effect. However, because
% of issues like this, it may be the safest practice to put all your
% \label just after \caption rather than within \caption{}.
%
% Reminder: the "draftcls" or "draftclsnofoot", not "draft", class
% option should be used if it is desired that the figures are to be
% displayed while in draft mode.
%
%\begin{figure}[!t]
%\centering
%\includegraphics[width=2.5in]{myfigure}
% where an .eps filename suffix will be assumed under latex, 
% and a .pdf suffix will be assumed for pdflatex; or what has been declared
% via \DeclareGraphicsExtensions.
%\caption{Simulation Results}
%\label{fig_sim}
%\end{figure}

% Note that IEEE typically puts floats only at the top, even when this
% results in a large percentage of a column being occupied by floats.


% An example of a double column floating figure using two subfigures.
% (The subfig.sty package must be loaded for this to work.)
% The subfigure \label commands are set within each subfloat command, the
% \label for the overall figure must come after \caption.
% \hfil must be used as a separator to get equal spacing.
% The subfigure.sty package works much the same way, except \subfigure is
% used instead of \subfloat.
%
%\begin{figure*}[!t]
%\centerline{\subfloat[Case I]\includegraphics[width=2.5in]{subfigcase1}%
%\label{fig_first_case}}
%\hfil
%\subfloat[Case II]{\includegraphics[width=2.5in]{subfigcase2}%
%\label{fig_second_case}}}
%\caption{Simulation results}
%\label{fig_sim}
%\end{figure*}
%
% Note that often IEEE papers with subfigures do not employ subfigure
% captions (using the optional argument to \subfloat), but instead will
% reference/describe all of them (a), (b), etc., within the main caption.


% An example of a floating table. Note that, for IEEE style tables, the 
% \caption command should come BEFORE the table. Table text will default to
% \footnotesize as IEEE normally uses this smaller font for tables.
% The \label must come after \caption as always.
%
%\begin{table}[!t]
%% increase table row spacing, adjust to taste
%\renewcommand{\arraystretch}{1.3}
% if using array.sty, it might be a good idea to tweak the value of
% \extrarowheight as needed to properly center the text within the cells
%\caption{An Example of a Table}
%\label{table_example}
%\centering
%% Some packages, such as MDW tools, offer better commands for making tables
%% than the plain LaTeX2e tabular which is used here.
%\begin{tabular}{|c||c|}
%\hline
%One & Two\\
%\hline
%Three & Four\\
%\hline
%\end{tabular}
%\end{table}


% Note that IEEE does not put floats in the very first column - or typically
% anywhere on the first page for that matter. Also, in-text middle ("here")
% positioning is not used. Most IEEE journals/conferences use top floats
% exclusively. Note that, LaTeX2e, unlike IEEE journals/conferences, places
% footnotes above bottom floats. This can be corrected via the \fnbelowfloat
% command of the stfloats package.




% trigger a \newpage just before the given reference
% number - used to balance the columns on the last page
% adjust value as needed - may need to be readjusted if
% the document is modified later
%\IEEEtriggeratref{8}
% The "triggered" command can be changed if desired:
%\IEEEtriggercmd{\enlargethispage{-5in}}

% references section

% can use a bibliography generated by BibTeX as a .bbl file
% BibTeX documentation can be easily obtained at:
% http://www.ctan.org/tex-archive/biblio/bibtex/contrib/doc/
% The IEEEtran BibTeX style support page is at:
% http://www.michaelshell.org/tex/ieeetran/bibtex/
%\bibliographystyle{IEEEtran}
% argument is your BibTeX string definitions and bibliography database(s)
%\bibliography{IEEEabrv,../bib/paper}
%
% <OR> manually copy in the resultant .bbl file
% set second argument of \begin to the number of references
% (used to reserve space for the reference number labels box)
%\begin{thebibliography}{1}

%\bibitem{IEEEhowto:kopka}
%H.~Kopka and P.~W. Daly, \emph{A Guide to \LaTeX}, 3rd~ed.\hskip 1em plus
%  0.5em minus 0.4em\relax Harlow, England: Addison-Wesley, 1999.

%\bibitem{youtube_gm/Id}

%\end{thebibliography}


% that's all folks
\end{document}


